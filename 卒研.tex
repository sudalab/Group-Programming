\documentclass[twocolumn,10pt,a4j]{ltjsarticle}
\usepackage{kougai}

\title{卒論梗概}
\author{2132151 吉田 将梧  指導教員 須田 宇宙 准教授}
\date{}

\begin{document}

\maketitle

\section{背景}
%1章には,背景・問題点・目的を順番に書く.
%背景は,広く一般的な事柄を書いて,読む人に同意を抱かせつつ問題点につなぐ.
%問題点では,「〜という問題点がある」などのように,「問題」または「問題点」と言う単語を用いて,目的につなぐ.
%目的では,「そこで本研究では」から始めて,「〜を目的とする」で締める.
%以下は過去の卒業研究最終審査用の梗概の抜粋である.

%背景
現在の情報化社会において,コンピュータやインターネットなどIT技術を活用する方法を学習する必要性が高まっている.さらに,2020年から小学校でのプログラミング教育が必修化された.この目的としては,プログラミング的思考力の育成,ITに興味を持ち積極的に関わる人材を育むことなどがあげられる.






%問題点
Web版のシミュレータ教材は,オフライン状態で利用できないことが問題点として挙げられる.
そこで,シミュレータ教材を電子書籍に搭載することで,オフライン状態でも利用できると考えた.
しかし,搭載して動作させる上で,シミュレータ教材の機能が制限される場合がある.
さらにシミュレータ教材をそのまま搭載しても,レイアウトの関係から利用しづらいという問題がある.
この問題に対して,音響教育関係者がノウハウを共有し,お互いの開発したシミュレータ教材をカスタマイズして電子書籍化することが望ましいと考えている.

%目的
そこで本研究では,音響教育関係者が容易に電子教科書を制作できるよう,シミュレータ教材を搭載した電子教科書のサンプルの制作と,その開発ガイドラインを作成し公開することを目的とする.

\section{内容}


\section{やること・やったこと}





\section{今後の予定}


\end{document}
