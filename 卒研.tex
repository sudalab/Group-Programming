\documentclass[twocolumn,10pt,a4j]{ltjsarticle}
\usepackage{kougai}

\title{卒論梗概}
\author{2132151 吉田 将梧  指導教員 須田 宇宙 准教授}
\date{}

\begin{document}

\maketitle

\section{背景}
%背景
近年,コンピュータは人々の生活の様々な場面で活用されている.スマートフォンやパソコン,自動車など情報化が加速している社会の中で不可欠なものとなっている.そこで,コンピュータをより適切かつ効果的に活用していくためにはその仕組みを知ることが重要である.その中の方法の一つとして「プログラミング」が挙げられる.

文部科学省は,初等・中等・高等教育においてプログラミング教育を行い日本人全体のICT活用力を底上げし,情報化社会に対応できる人材を育むことを目標にしている.

%問題点
しかし,現在の大学でのプログラミング教育では,IT業界で行われているような複数人でのシステムの開発の経験があまりなく,社会に出た後と授業で行われている内容にギャップが生じてしまっている.

そこで,IT業界のようにプログラミング学習をグループで行うことにより,システム開発の経験としてギャップが生まれにくくなると考える.
グループでのプログラミング学習の有用性は,先行研究によりプログラムの理解や満足感について効果的であると報告されている.

しかし,グループでのプログラミング学習には評価方法が曖昧であり,個人での頑張りが評価されにくいという問題点がある.

%目的
そこで本研究では,グループでのプログラミング学習における個人貢献度を推定し,グループの成果物だけの評価ではなく個人の評価を可能にすることが目的とする.

\section{現在のプログラミング教育の問題点}
プログラミング教育において,

\section{調査方法}





\section{今後の予定}
グループでのプログラミング学習を行った際に,グループとしての成果物だけを評価するのではなく,個人の貢献度を可視化するために
さらにそこから個人貢献度を推定するためにはどのような条件があるのかなどを調査・検証する.

\end{document}
